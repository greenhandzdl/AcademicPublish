% documentclass[a4paper,12pt]{article}
\documentclass[a4paper,12pt]{ctexart}
\usepackage{amsmath, amssymb}
\usepackage{graphicx}
\usepackage{hyperref}


\begin{document}

\title{二项式定理的组合数表示}
\author{greenhandzdl}
\date{\today}
\maketitle

\pagenumbering{roman}
\tableofcontents
\newpage
\pagenumbering{arabic}

\section{预备知识}
\noindent 已知泰勒展开式:
\begin{eqnarray}
f(x)=f(x_{0})+\frac{f^{\prime}(x_{0})}{1!}(x-x_{0})+\frac{f^{\prime\prime}(x_{0})}{2!}(x-x_{0})^{2}+\cdots+\frac{f^{(n)}(x_{0})}{n!}(x-x_{0})^{n}+R_{n} \\
R_{n}(x)=o[(x-x_{0})^{n}]
\end{eqnarray}
自然,我们可以推导出$(1+x)^\alpha$的展开式:
\begin{equation}
=1+\alpha x+\frac{\alpha(\alpha-1)}{2}x^2+o(x^2)
\end{equation}
由于\textbf{广义组合数}的存在
\begin{eqnarray}
m \in \mathbb{R},n \in \mathbb{Q},C_m^n=\frac{\prod_{i=m-n+1}^mi}{n!} \\
C_{-m}^{n}=\frac{\prod_{i=-m-n+1}^{-m}i}{n!}=\frac{(-1)^n(m+n-1)\cdots(m+1)m}{n!}=(-1)^nC_{m+n-1}^{m-1}.
\end{eqnarray}
然后,进行一些化简,就得到主角(似乎与中学阶段学过的二项式定理相符):
\begin{equation}
(1+x)^n = \sum_{i=0}^{+ \infty }  \tbinom{n}{i} \cdot x^i
\end{equation}

\section{推广}
\paragraph{因此,既然证明了组合数表示的等价,那么,不妨来寻找某个数n可以使展开式在某一项正负交替.}
这里以$ n= \frac{1}{2}  $举例,不妨代入式子的形式(注意:\href{https://zhuanlan.zhihu.com/p/358479652}{复制粘贴的例3}):
\begin{equation}
\begin{aligned}(1+x)^{\frac{1}{2}}
	&\begin{aligned}&=1+\sum_{n=1}^\infty\frac{\frac{1}{2}(\frac{1}{2}-1)\cdots(\frac{1}{2}-n+1)}{n!}x^n\end{aligned}\\
	&=1+\frac{1}{2}x+\sum_{n=2}^\infty\frac{(-1)^{n-1}\cdot1\cdot3\cdots(2n-3)}{2^n\cdot n!}x^n\\
	&=1+\frac{1}{2}x+\sum_{n=2}^\infty\frac{(-1)^{n-1}}{(2n-1)\cdot4^n}C_{2n}^nx^n\\
	&=\sum_{n=0}^\infty\frac{(-1)^{n-1}}{(2n-1)\cdot4^n}C_{2n}^nx^n.\end{aligned}
\end{equation}
显然,可以看出$(1+x)^n$当$n \not \in \mathbb{Z} $时,展开式会在某一项后面出现正负交替形式.


\end{document}